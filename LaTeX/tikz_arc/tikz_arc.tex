\documentclass{amsart}

% Source: https://github.com/8dcc/scratch/blob/main/LaTeX/tikz_arc/tikz_arc.tex
\title{Arcs in TikZ}
\author{8dcc}

% Tikz for graphs, Calc for coordinate calculations, and Decorations for drawing
% the braces.
\usepackage{tikz}
\usetikzlibrary{calc}
\usetikzlibrary{decorations.pathreplacing}

% Code highlighting.
\usepackage{listings}
\lstset{
  % Showing spaces
  showspaces=false,
  showstringspaces=false,
  showtabs=false,
  % Indentation and breaks
  tabsize=4,
  breaklines=true,
  breakatwhitespace=true,
  columns=flexible,
  % Show left, right, top and bottom borders
  frame=tblr,
  % Misc
  aboveskip=3mm,
  belowskip=3mm,
  basicstyle={\small\ttfamily},
}

% Different monospace font for code blocks (listings)
\usepackage{inconsolata}

% Hyperlinks, for the bibliography
\usepackage[hidelinks]{hyperref}

% Remove author and extra info from the headers
\pagestyle{plain}

% Change the spacing between paragraphs
\setlength{\parskip}{\baselineskip}

% New environment adding spacing for tikz pictures
\newenvironment{tikzpicturecenter}
{\begin{center}\begin{tikzpicture}}
{\end{tikzpicture}\end{center}}

% Fancy TikZ text, like \LaTeX
\newcommand{\TikZ}{Ti\textit{k}Z}

\begin{document}
\maketitle

When I started reading about arcs in \TikZ, the \LaTeX\ graphics library, I had
some trouble understanding how they worked. This short explanation (made with
\TikZ\ itself) might help someone who is looking for more information.

\section{Syntax}

A \TikZ\ arc is just a portion of a circumference that starts at an angle
$\alpha$ and ends at an angle $\beta$.

\begin{tikzpicturecenter}
  % Radius of the circumference.
  \pgfmathsetmacro{\Radius}{2}

  % Angle of the Alpha and Beta arcs.
  \pgfmathsetmacro{\ArcAngleAlpha}{15}
  \pgfmathsetmacro{\ArcAngleBeta}{40}

  % Point in the circumference where the Alpha arc will end, and the Beta arc
  % will start. Based on the Alpha angle and circumference radius.
  \pgfmathsetmacro{\AlphaX}{\Radius*cos(\ArcAngleAlpha)}
  \pgfmathsetmacro{\AlphaY}{\Radius*sin(\ArcAngleAlpha)}

  % Point in the circumference where the Beta arc will start. Same as before,
  % but we accumulate the angle of Alpha.
  \pgfmathsetmacro{\BetaX}{\Radius*cos(\ArcAngleAlpha+\ArcAngleBeta)}
  \pgfmathsetmacro{\BetaY}{\Radius*sin(\ArcAngleAlpha+\ArcAngleBeta)}

  % Calculate the middle point of the Alpha and Beta arcs. In other words,
  % calculate the end point of an arc of half the angles.
  \pgfmathsetmacro{\AlphaLabelX}{\Radius*cos(\ArcAngleAlpha/2)}
  \pgfmathsetmacro{\AlphaLabelY}{\Radius*sin(\ArcAngleAlpha/2)}
  \pgfmathsetmacro{\BetaLabelX}{\Radius*cos(\ArcAngleAlpha+(\ArcAngleBeta/2))}
  \pgfmathsetmacro{\BetaLabelY}{\Radius*sin(\ArcAngleAlpha+(\ArcAngleBeta/2))}

  % Origin: Center of the circle.
  \coordinate (o) at (0,0);
  \coordinate (a) at (\AlphaX, \AlphaY);
  \coordinate (b) at (\BetaX, \BetaY);

  % Draw horizontal line.
  \draw[-stealth] (o) -- (2.3,0);
  \draw[blue] (o) -- (a);
  \draw[red] (o) -- (b);

  % Draw circle and center.
  \draw[dotted] (o) circle (\Radius);
  \filldraw (o) circle (1pt);

  % Draw the Alpha arc.
  \draw[thick, blue] (\Radius, 0) arc(0:\ArcAngleAlpha:\Radius);

  % Draw the Beta arc, with alternative syntax.
  \draw[very thick, red] (a) arc[start angle=\ArcAngleAlpha, end angle=\ArcAngleAlpha + \ArcAngleBeta, radius=\Radius];

  % Draw the Alpha and Beta labels.
  \node at (\AlphaLabelX, \AlphaLabelY) [left, blue] {$\alpha$};
  \filldraw (a) circle (1pt) node[above right]{$A$};
  \node at (\BetaLabelX, \BetaLabelY) [below left, red] {$\beta$};
\end{tikzpicturecenter}

Where the red arc, representing the angle $\beta$, is what we are trying to
draw. The arc starts at point $A$, which itself is at an angle $\alpha$ relative
to the positive horizontal.

These are the two main ways of declaring an arc in \TikZ, both are equivalent:

\begin{lstlisting}[language=tex]
  \draw (X,Y) arc (START:END:RADIUS);
  \draw (X,Y) arc [start angle=START, end angle=END, radius=RADIUS];
\end{lstlisting}

The $(X,Y)$ point does not represent the center of the circumference, but the
point where the arc itself starts, which was named $A$ in the previous example.

You also need to provide the \textit{start} angle of the arc. This angle, in
degrees, is relative to the positive horizontal axis. In the previous example,
it was the blue angle, named $\alpha$. The \textit{end} angle indicates where
the arc will end on the circumference, and is \textbf{not} relative to the start
angle $\alpha$, but relative to the positive horizontal axis. In other words, it
expects an angle relative to the horizontal, instead of the angle of the arc we
are trying to draw.

Finally, it expects the \textit{radius} of the circumference. That is, the
distance between the \textit{start point} and the center of the ``hypothetical''
circumference where the arc is. If you are, for example, drawing an arc to
represent the angle between two lines, the radius will probably be the distance
between the start of your arc and the vertex where the lines join, which would
be the center of this ``circumference''.

Note how we didn't specify the center of the circumference ourselves, but the
arc function will calculate it with the \textit{start point}, \textit{start
  angle} and \textit{radius} values. Specifically, it calculates the center of
the circumference with the following formula:

\begin{align*}
C_x &= A_x - r \cos \alpha \\
C_y &= A_y - r \sin \alpha
\end{align*}

Therefore, we can also calculate the $A$ (and optionally $B$) positions if we
already know the \textit{center}, \textit{radius} and \textit{start angle} with
the following formula:

\begin{align*}
A_x &= C_x + r \cos \alpha \\
A_y &= C_y + r \sin \alpha
\end{align*}

If you are not sure why this works, the classic $\sin$ and $\cos$ graph might
help:

\begin{tikzpicturecenter}
  \pgfmathsetmacro{\Radius}{2}
  \pgfmathsetmacro{\ArcRadius}{1}

  \pgfmathsetmacro{\ArcAngleAlpha}{40}

  \pgfmathsetmacro{\AlphaX}{\ArcRadius*cos(\ArcAngleAlpha)}
  \pgfmathsetmacro{\AlphaY}{\ArcRadius*sin(\ArcAngleAlpha)}

  \pgfmathsetmacro{\LabelX}{\ArcRadius*cos(\ArcAngleAlpha/2)}
  \pgfmathsetmacro{\LabelY}{\ArcRadius*sin(\ArcAngleAlpha/2)}

  \coordinate (o) at (0,0);
  \coordinate (ux) at (\Radius,0);
  \coordinate (uy) at (0,\Radius);
  \coordinate (ua) at (\AlphaX, \AlphaY);
  \coordinate (x) at ($1.1*(ux)$);
  \coordinate (y) at ($1.1*(uy)$);
  \coordinate (a) at ($\Radius*(ua)$);

  \filldraw[violet, fill opacity=0.3] (o) -- (\ArcRadius,0) arc (0:\ArcAngleAlpha:\ArcRadius) -- cycle;
  \node at (\LabelX, \LabelY) [left] {$\theta$};

  \draw[dotted] (o) circle (\Radius);

  \draw[-stealth, gray] (o) -- (x);
  \draw[-stealth, gray] (o) -- (y);
  \draw[thick, red]  (o) -- ($(\Radius*\AlphaX, 0)$) node[below, pos=0.5]{$\cos(\theta)$};
  \draw[thick, blue] (a) -- ($(\Radius*\AlphaX, 0)$) node[right, pos=0.5]{$\sin(\theta)$};
  \draw[thick] (o) -- (a) node[yshift=0.3cm, xshift=-0.3cm, pos=0.6]{$r=1$};
  \filldraw[gray] (o) circle (1pt);
\end{tikzpicturecenter}

\newpage

\section{Example}

Let's look at a more specific example of how an arc would be drawn. Imagine we
wanted to draw the arc representing the $\beta$ angle formed by the $A$ and $B$
lines.

\begin{tikzpicturecenter}
  % For details on un-commented sections, see previous TikZ picture.

  \pgfmathsetmacro{\Radius}{2}

  % Center of the circumference. Not needed for this example.
  \pgfmathsetmacro{\OriginX}{0}
  \pgfmathsetmacro{\OriginY}{0}

  \pgfmathsetmacro{\ArcAngleAlpha}{15}
  \pgfmathsetmacro{\ArcAngleBeta}{40}

  \pgfmathsetmacro{\AlphaX}{\OriginX+\Radius*cos(\ArcAngleAlpha)}
  \pgfmathsetmacro{\AlphaY}{\OriginY+\Radius*sin(\ArcAngleAlpha)}

  \pgfmathsetmacro{\BetaX}{\OriginX+\Radius*cos(\ArcAngleAlpha+\ArcAngleBeta)}
  \pgfmathsetmacro{\BetaY}{\OriginY+\Radius*sin(\ArcAngleAlpha+\ArcAngleBeta)}

  % Magnitudes of A, B and H vectors. Will be used for getting the end of the
  % lines by multiplying by the unit vectors.
  \pgfmathsetmacro{\LineMagnitude}{1.75}

  \pgfmathsetmacro{\AlphaLabelX}{\OriginX+\Radius*cos(\ArcAngleAlpha/2)}
  \pgfmathsetmacro{\AlphaLabelY}{\OriginY+\Radius*sin(\ArcAngleAlpha/2)}
  \pgfmathsetmacro{\BetaLabelX}{\OriginX+\Radius*cos(\ArcAngleAlpha+(\ArcAngleBeta/2))}
  \pgfmathsetmacro{\BetaLabelY}{\OriginY+\Radius*sin(\ArcAngleAlpha+(\ArcAngleBeta/2))}

  % The UA and UB coordinates represent where the arc starts and ends, while the
  % A and B coordinates represent where the lines themselves will end.
  \coordinate (o) at (\OriginX, \OriginY);
  \coordinate (ua) at (\AlphaX, \AlphaY);
  \coordinate (ub) at (\BetaX, \BetaY);
  \coordinate (a) at ($\LineMagnitude*(ua)$);
  \coordinate (b) at ($\LineMagnitude*(ub)$);

  % Draw dotted circle itself.
  \draw[dotted] (o) circle (\Radius);

  % The Alpha arc starts to the left of the Origin, at Radius units.
  \draw[thick, blue] ($(o) + (\Radius, 0)$) arc[start angle=0, end angle=\ArcAngleAlpha, radius=\Radius];
  \draw[thick, red] (ua) arc[start angle=\ArcAngleAlpha, end angle=\ArcAngleAlpha + \ArcAngleBeta, radius=\Radius];

  % The brace to indicate the length of the Radius.
  \draw[decorate, decoration={brace,amplitude=5pt,mirror}] (o) -- ($(o) + (\Radius, 0)$) node [midway, yshift=-0.4cm] {$r$};

  % Horizontal line from Origin, with the same length as the A and B lines.
  \draw[-stealth, gray] (o) -- ($(o) + (\LineMagnitude*\Radius, 0)$) node[below, pos=0.8]{$x$};

  % The A and B lines.
  \draw[-stealth] (o) -- (a) node[above]{$A$};
  \draw[-stealth] (o) -- (b) node[above]{$B$};

  % The Alpha and Beta arcs.
  \node at (\AlphaLabelX, \AlphaLabelY) [left, blue] {$\alpha$};
  \node at (\BetaLabelX, \BetaLabelY) [below left, red] {$\beta$};

  % The center of the circumference, also the vertex of A and B.
  \filldraw (o) circle (1pt) node[below left]{$c$};

  % The points where the Alpha and Beta arcs end.
  \filldraw (\AlphaX, \AlphaY) circle (1pt) node [xshift=0.2cm, yshift=0.35cm] {$A_1$};
  \filldraw (\BetaX, \BetaY) circle (1pt) node [xshift=-0.1cm, yshift=0.35cm] {$B_1$};
\end{tikzpicturecenter}

Note that the arc will be drawn from $A_1$ to $B_1$, not the other way around.

We should know, at least, the distance from the center $c$ where the arc will be
drawn inside the line $A$. This is the radius $r$ of the circumference. Perhaps
we don't know specifically where the $A_1$ point will be, but we can calculate
it with some simple trigonometry:

\begin{lstlisting}[language=tex]
  % Radius of the circumference.
  \pgfmathsetmacro{\Radius}{2}

  % Center of the circumference, vertex where A and B join.
  \pgfmathsetmacro{\OriginX}{0}
  \pgfmathsetmacro{\OriginY}{0}

  % Angle in degrees where the arc Alpha ends.
  \pgfmathsetmacro{\ArcAngleAlpha}{15}

  % Calculate A1 point, from the Center, Radius and Angle.
  \pgfmathsetmacro{\AlphaX}{\OriginX+\Radius*cos(\ArcAngleAlpha)}
  \pgfmathsetmacro{\AlphaY}{\OriginY+\Radius*sin(\ArcAngleAlpha)}

  % Define the coordinate for the A1 point, for convenience.
  \coordinate (a1) at (\AlphaX, \AlphaY);
\end{lstlisting}

\newpage

With that, we can draw our arc using the vertex as the center of the
``circumference'':

\begin{lstlisting}[language=tex]
  % Angle in degrees where the arc Beta ends. Relative to `x'.
  \pgfmathsetmacro{\ArcAngleBeta}{40}

  % Draw the arc itself.
  \draw[thick, red] (a1)
      arc[start angle=\ArcAngleAlpha,
          end angle=\ArcAngleAlpha + \ArcAngleBeta,
          radius=\Radius];
\end{lstlisting}

We are drawing a thick red arc, starting at point $A_1$, which is at 15 degrees
in a hypotetical circumference of radius 2. The arc ends at 40 degrees relative
to $x$.

\section{Filling the arc}

If we wanted to fill the arc, we would need to use \verb|\filldraw| instead of
\verb|\draw|, and we would have to manually specify the origin of our
``circumference''. We would also need to add \verb|cycle| at the end:


\begin{lstlisting}[language=tex]
  % Without filling
  \draw[blue] (a1) arc (15:55:2);

  % With filling
  \filldraw[blue, fill opacity=0.3] (o) -- (a1) arc (15:55:2) -- cycle;
\end{lstlisting}

This is the result:

\begin{tikzpicturecenter}
  \pgfmathsetmacro{\Radius}{1.5}

  \pgfmathsetmacro{\OriginX}{0}
  \pgfmathsetmacro{\OriginY}{0}

  \pgfmathsetmacro{\ArcAngleAlpha}{10}
  \pgfmathsetmacro{\ArcAngleBeta}{110}

  \pgfmathsetmacro{\AlphaX}{\OriginX+\Radius*cos(\ArcAngleAlpha)}
  \pgfmathsetmacro{\AlphaY}{\OriginY+\Radius*sin(\ArcAngleAlpha)}
  \pgfmathsetmacro{\BetaX}{\OriginX+\Radius*cos(\ArcAngleAlpha+\ArcAngleBeta)}
  \pgfmathsetmacro{\BetaY}{\OriginY+\Radius*sin(\ArcAngleAlpha+\ArcAngleBeta)}

  \pgfmathsetmacro{\LineMagnitude}{1.5}

  \pgfmathsetmacro{\LabelX}{\OriginX+\Radius*cos(\ArcAngleAlpha+(\ArcAngleBeta/2))}
  \pgfmathsetmacro{\LabelY}{\OriginY+\Radius*sin(\ArcAngleAlpha+(\ArcAngleBeta/2))}

  \coordinate (o) at (\OriginX, \OriginY);
  \coordinate (ua) at (\AlphaX, \AlphaY);
  \coordinate (ub) at (\BetaX, \BetaY);
  \coordinate (a) at ($\LineMagnitude*(ua)$);
  \coordinate (b) at ($\LineMagnitude*(ub)$);

  % Fill the arc between O, A1 and B1.
  \filldraw[blue, fill opacity=0.3] (o) -- (ua) arc (\ArcAngleAlpha:\ArcAngleAlpha+\ArcAngleBeta:\Radius) -- cycle;

  \node at (\LabelX, \LabelY) [below left] {$\theta$};

  % Lines and dot
  \draw[-stealth, thick] (o) -- (a) node[above]{$A$};
  \draw[-stealth, thick] (o) -- (b) node[above]{$B$};
  \filldraw (o) circle (1pt);
\end{tikzpicturecenter}

\newpage

\section{Getting the position for the labels}

To get the positions where the $\alpha$, $\beta$ and $\theta$ labels will go,
you just need to calculate the point in the middle of the arc. Just divide the
angle of the arc by two, and calculate the point like we did before. This is the
code used for getting the labels of section 2:

\begin{lstlisting}[language=tex]
  \pgfmathsetmacro{\AlphaLabelX}{\OriginX+\Radius*cos(\ArcAngleAlpha/2)}
  \pgfmathsetmacro{\AlphaLabelY}{\OriginY+\Radius*sin(\ArcAngleAlpha/2)}

  \pgfmathsetmacro{\BetaLabelX}{\OriginX+\Radius*cos(\ArcAngleAlpha+(\ArcAngleBeta/2))}
  \pgfmathsetmacro{\BetaLabelY}{\OriginY+\Radius*sin(\ArcAngleAlpha+(\ArcAngleBeta/2))}
\end{lstlisting}

In the following image, the dots in black represent the points for the
labels. The actual label text would be calculated relative to that.

\begin{tikzpicturecenter}
  % For details on un-commented sections, see previous TikZ picture.

  \pgfmathsetmacro{\Radius}{1.5}

  \pgfmathsetmacro{\OriginX}{0}
  \pgfmathsetmacro{\OriginY}{0}

  \pgfmathsetmacro{\ArcAngleAlpha}{15}
  \pgfmathsetmacro{\ArcAngleBeta}{40}

  \pgfmathsetmacro{\AlphaX}{\OriginX+\Radius*cos(\ArcAngleAlpha)}
  \pgfmathsetmacro{\AlphaY}{\OriginY+\Radius*sin(\ArcAngleAlpha)}

  \pgfmathsetmacro{\BetaX}{\OriginX+\Radius*cos(\ArcAngleAlpha+\ArcAngleBeta)}
  \pgfmathsetmacro{\BetaY}{\OriginY+\Radius*sin(\ArcAngleAlpha+\ArcAngleBeta)}

  \pgfmathsetmacro{\LineMagnitude}{1.5}

  \pgfmathsetmacro{\AlphaLabelX}{\OriginX+\Radius*cos(\ArcAngleAlpha/2)}
  \pgfmathsetmacro{\AlphaLabelY}{\OriginY+\Radius*sin(\ArcAngleAlpha/2)}
  \pgfmathsetmacro{\BetaLabelX}{\OriginX+\Radius*cos(\ArcAngleAlpha+(\ArcAngleBeta/2))}
  \pgfmathsetmacro{\BetaLabelY}{\OriginY+\Radius*sin(\ArcAngleAlpha+(\ArcAngleBeta/2))}

  \coordinate (o) at (\OriginX, \OriginY);
  \coordinate (ua) at (\AlphaX, \AlphaY);
  \coordinate (ub) at (\BetaX, \BetaY);
  \coordinate (a) at ($\LineMagnitude*(ua)$);
  \coordinate (b) at ($\LineMagnitude*(ub)$);

  \draw[thick, blue] ($(o) + (\Radius, 0)$) arc[start angle=0, end angle=\ArcAngleAlpha, radius=\Radius];
  \draw[thick, red] (ua) arc[start angle=\ArcAngleAlpha, end angle=\ArcAngleAlpha + \ArcAngleBeta, radius=\Radius];

  % Draw points where the Alpha and Beta labels would go.
  \filldraw (\AlphaLabelX, \AlphaLabelY) circle (1pt);
  \filldraw (\BetaLabelX, \BetaLabelY) circle (1pt);

  \draw[-stealth, gray] (o) -- (a) node[above]{$A$};
  \draw[-stealth, gray] (o) -- (b) node[below right]{$B$};
  \draw[-stealth, gray] (o) -- ($(o) + (\LineMagnitude*\Radius, 0)$) node[below]{$x$};
\end{tikzpicturecenter}

\begin{thebibliography}{9}
\bibitem{manual}
  TikZ and PGF Manual for version 1.18, 2007. Section 11.8, The Arc Operation.

  \url{https://www.bu.edu/math/files/2013/08/tikzpgfmanual.pdf}
\bibitem{tutorial}
  \url{https://tikz.dev/tutorial}
\bibitem{definearc}
  \url{https://tex.stackexchange.com/q/175016/292826}
\bibitem{drawarc}
  \url{https://tex.stackexchange.com/q/54142/292826}
\bibitem{fillarc}
  \url{https://tex.stackexchange.com/q/62128/292826}
\end{thebibliography}

\end{document}
