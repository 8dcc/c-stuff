\documentclass{amsart}

\title{Tikz Arc}
\author{8dcc}

% Graphs, with coordinate calculations
\usepackage{tikz}
\usetikzlibrary{calc}

% Code highlighting
\usepackage{listings}
\lstset{frame=tblr,
  aboveskip=3mm,
  belowskip=3mm,
  showstringspaces=false,
  columns=flexible,
  basicstyle={\small\ttfamily},
  numbers=none,
  breaklines=true,
  breakatwhitespace=true,
  tabsize=4
}

% Hyperlinks, for the bibliography
\usepackage[hidelinks]{hyperref}

% Remove author and extra info from the headers
\pagestyle{plain}

% Change the spacing between paragraphs
\setlength{\parskip}{\baselineskip}

% New environment adding spacing for tikz pictures
\newenvironment{tikzpicturecenter}
{\begin{center}\begin{tikzpicture}}
{\end{tikzpicture}\end{center}}

\begin{document}
\maketitle

When I started reading about arcs in TikZ, I had some trouble understanding
them. This short explanation might help someone who is looking for more
information.

A TikZ arc is just a portion of a circumference that starts at an angle $\alpha$
and ends at an angle $\beta$.

\begin{tikzpicturecenter}
  % Radius of the circumference.
  \pgfmathsetmacro{\Radius}{2}

  % Angle of the Alpha and Beta arcs.
  \pgfmathsetmacro{\ArcAngleAlpha}{15}
  \pgfmathsetmacro{\ArcAngleBeta}{40}

  % Point in the circumference where the Alpha arc will end, and the Beta arc
  % will start. Based on the Alpha angle and circumference radius.
  \pgfmathsetmacro{\AlphaX}{\Radius*cos(\ArcAngleAlpha)}
  \pgfmathsetmacro{\AlphaY}{\Radius*sin(\ArcAngleAlpha)}

  % Point in the circumference where the Beta arc will start. Same as before,
  % but we accumulate the angle of Alpha.
  \pgfmathsetmacro{\BetaX}{\Radius*cos(\ArcAngleAlpha+\ArcAngleBeta)}
  \pgfmathsetmacro{\BetaY}{\Radius*sin(\ArcAngleAlpha+\ArcAngleBeta)}

  % Calculate the middle point of the Alpha and Beta arcs. In other words,
  % calculate the end point of an arc of half the angles.
  \pgfmathsetmacro{\AlphaLabelX}{\Radius*cos(\ArcAngleAlpha/2)}
  \pgfmathsetmacro{\AlphaLabelY}{\Radius*sin(\ArcAngleAlpha/2)}
  \pgfmathsetmacro{\BetaLabelX}{\Radius*cos(\ArcAngleAlpha+(\ArcAngleBeta/2))}
  \pgfmathsetmacro{\BetaLabelY}{\Radius*sin(\ArcAngleAlpha+(\ArcAngleBeta/2))}

  % Origin: Center of the circle.
  \coordinate (o) at (0,0);
  \coordinate (a) at (\AlphaX, \AlphaY);
  \coordinate (b) at (\BetaX, \BetaY);

  % Draw horizontal line
  \draw[-stealth] (o) -- (2.3,0);
  \draw[blue] (o) -- (a);
  \draw[red] (o) -- (b);

  % Draw circle and center.
  \draw[dotted] (o) circle (\Radius);
  \filldraw (o) circle (1pt);

  % Draw the Alpha arc.
  \draw[thick, blue] (\Radius, 0) arc(0:\ArcAngleAlpha:\Radius);

  % Draw the Beta arc, with alternative syntax.
  \draw[very thick, red] (a) arc[start angle=\ArcAngleAlpha, end angle=\ArcAngleAlpha + \ArcAngleBeta, radius=\Radius];

  % Draw the Alpha and Beta labels.
  \node at (\AlphaLabelX, \AlphaLabelY) [left, blue] {$\alpha$};
  \filldraw (a) circle (1pt) node[above right]{$A$};
  \node at (\BetaLabelX, \BetaLabelY) [below left, red] {$\beta$};
\end{tikzpicturecenter}

Where the red arc for the angle $\beta$ is what we are trying to draw. The arc
starts at point $A$, which itself is at an angle $\alpha$ relative to the
positive horizontal.

These are the two main ways of declaring an arc in TikZ, both are equivalent:

\begin{lstlisting}[language=tex]
  \draw (X,Y) arc (START:END:RADIUS);
  \draw (X,Y) arc [start angle=START, end angle=END, radius=RADIUS];
\end{lstlisting}

The $(X,Y)$ point does not represent the center of the circumference, but the
point where the arc itself starts, which was named $A$ in the previous example.

You also need to provide the \textit{start} angle of the arc. This angle, in
degrees, is relative to the positive horizontal axis. In the previous example,
it was the blue angle, named $\alpha$. The \textit{end} angle indicates where
the arc will end on the circumference, and is \textbf{not} relative to the start
angle $\alpha$, but relative to the positive horizontal axis. In other words, it
expects an angle relative to the horizontal, instead of the angle of the arc we
are trying to draw.

Finally, it expects the \textit{radius} of the circumference. If you are, for
example, drawing an arc to represent the angle between two lines, the radius
will probably be the distance between the start of your arc and the point where
the lines join, which will be the center of this ``circumference''.

Note how we didn't specify the center of the circumference ourselves, but the
arc function will calculate it with the \textit{start point}, \textit{start
  angle} and \textit{radius} values. Specifically, it calculates the center of
the circumference with the following formula:

\begin{align*}
C_x &= A_x - r \cos \alpha \\
C_y &= A_y - r \sin \alpha
\end{align*}

Therefore, we can also calculate the $A$ (and optionally $B$) positions if we
already know the \textit{center}, \textit{radius} and \textit{start angle} with
the following formula:

\begin{align*}
A_x &= C_x + r \cos \alpha \\
A_y &= C_y + r \sin \alpha
\end{align*}

% TODO: Draw sine and cosine graphs here

Let's look at a more specific example of how the arc would be drawn.

% TODO: Draw specific example

\begin{tikzpicturecenter}
  % For details on un-commented sections, see previous TikZ picture.

  \pgfmathsetmacro{\Radius}{2}

  % Center of the circumference. Not needed for this example.
  \pgfmathsetmacro{\OriginX}{0}%
  \pgfmathsetmacro{\OriginY}{0}%

  \pgfmathsetmacro{\ArcAngleAlpha}{15}
  \pgfmathsetmacro{\ArcAngleBeta}{40}

  \pgfmathsetmacro{\AlphaX}{\OriginX+\Radius*cos(\ArcAngleAlpha)}
  \pgfmathsetmacro{\AlphaY}{\OriginY+\Radius*sin(\ArcAngleAlpha)}

  \pgfmathsetmacro{\BetaX}{\OriginX+\Radius*cos(\ArcAngleAlpha+\ArcAngleBeta)}
  \pgfmathsetmacro{\BetaY}{\OriginY+\Radius*sin(\ArcAngleAlpha+\ArcAngleBeta)}

  \pgfmathsetmacro{\AlphaLabelX}{\OriginX+\Radius*cos(\ArcAngleAlpha/2)}
  \pgfmathsetmacro{\AlphaLabelY}{\OriginY+\Radius*sin(\ArcAngleAlpha/2)}
  \pgfmathsetmacro{\BetaLabelX}{\OriginX+\Radius*cos(\ArcAngleAlpha+(\ArcAngleBeta/2))}
  \pgfmathsetmacro{\BetaLabelY}{\OriginY+\Radius*sin(\ArcAngleAlpha+(\ArcAngleBeta/2))}

  \coordinate (o) at (\OriginX, \OriginY);
  \coordinate (a) at (\AlphaX, \AlphaY);
  \coordinate (b) at (\BetaX, \BetaY);

  % Horizontal line from Origin, with length Radius+0.3 units.
  \draw[-stealth] (o) -- ($(\OriginX+\Radius+0.3, 0)$);
  \draw[blue] (o) -- (a);
  \draw[red] (o) -- (b);

  \draw[dotted] (o) circle (\Radius);
  \filldraw (o) circle (1pt);

  % The Alpha arc starts to the left of the Origin, at Radius units.
  \draw[thick, blue] ($(\OriginX+\Radius, 0)$) arc(0:\ArcAngleAlpha:\Radius);

  \draw[very thick, red] (a) arc[start angle=\ArcAngleAlpha, end angle=\ArcAngleAlpha + \ArcAngleBeta, radius=\Radius];

  \node at (\AlphaLabelX, \AlphaLabelY) [left, blue] {$\alpha$};
  \filldraw (a) circle (1pt) node[above right]{$A$};
  \node at (\BetaLabelX, \BetaLabelY) [below left, red] {$\beta$};

  % DELME
  \node at (-2, 2) {\textbf{WIP}};
\end{tikzpicturecenter}

\begin{thebibliography}{9}
\bibitem{definearc}
  \url{https://tex.stackexchange.com/q/175016/292826}
\bibitem{drawarc}
  \url{https://tex.stackexchange.com/q/54142/292826}
\bibitem{fillarc}
  \url{https://tex.stackexchange.com/q/62128/292826}
\end{thebibliography}

\end{document}
