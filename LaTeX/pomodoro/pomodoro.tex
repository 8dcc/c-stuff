\documentclass{amsart}

\title{Pomodoro intervals}
\author{8dcc}

% Various math utilities, like align*
\usepackage{amsmath}

% Graphs
\usepackage{tikz}

% Link sections and subsections
\usepackage[hidelinks]{hyperref}
\hypersetup{linktoc=all}

% Code highlighting.
\usepackage{listings}
\lstset{
  % Showing spaces
  showspaces=false,
  showstringspaces=false,
  showtabs=false,
  % Indentation and breaks
  tabsize=4,
  breaklines=true,
  breakatwhitespace=true,
  columns=flexible,
  % Show left, right, top and bottom borders
  frame=tblr,
  % Misc
  aboveskip=3mm,
  belowskip=3mm,
  basicstyle={\small\ttfamily},
}

% Different monospace font for code blocks (listings)
\usepackage{inconsolata}

% Remove author and extra info from the headers
\pagestyle{plain}

% New environment adding spacing for tikz pictures
\newenvironment{tikzpicturecenter}
{\begin{center}\begin{tikzpicture}}
    {\end{tikzpicture}\end{center}}

\begin{document}
\maketitle

\section{The Pomodoro Technique}

The Pomodoro Technique is a time management method developed by Francesco
Cirillo in the late 1980s. It uses a kitchen timer to break work into intervals
(\textit{pomodoros}), typically 25 minutes in length, separated by short
breaks. After 4 pomodoros are done, a bigger break is taken.

For this example, these will be the specifics timings:

\begin{enumerate}
\item Work for 25 minutes.
\item Rest for 5 minutes.
\item Go to step one, until 4 pomodoros have passed.
\item Once the 4 pomodoros are done, rest for 30 minutes.
\end{enumerate}

With this, we know that the first short break will start at minute 25 and end at
minute 30, and the first big break will start at minute 115 ($30 \cdot 4 - 5$)
and end at minute 140.

\section{The problem}

Given a number of minutes, how do we know if we should be resting or working
when that time has passed?

You can calculate if a minute is in a short or long break by performing a
modulus with the end of the break, and checking if that value is greater or
equal than the start of the break.

\begin{align*}
  \text{inShortBreak}(m) &= m \bmod 30 \geq 25 \\
  \text{inLongBreak}(m)  &= m \bmod 140 \geq 115
\end{align*}

However, the fifth pomodoro, the one after the first long break doesn't start
aligned to the 30 minute boundary ($140 \bmod 30 \neq 0$).

At first I thought about calculating the Least Common Multiple (LCM) of 140 and
30 (which is 420), but that won't work here.

\section{The solution}

The solution was actually quite simple. You can think of the break cycle as a
big 145-minute cycle, and calculate the minutes relative to that, just like we
did before.

\begin{align*}
  \text{inShortBreak}(m) &= \left( m \bmod 145 \right) \bmod 30 \geq 25 \\
  \text{inLongBreak}(m)  &= \left( m \bmod 145 \right) \bmod 140 \geq 115
\end{align*}

Or, in C:

\begin{lstlisting}[language=C]
bool inPomodoro(int minutes) {
    int minutes_in_clycle = minutes % ((25 + 5) * 3 + 25 + 30);
    bool in_long_rest = (minutes_in_clycle % 140 >= 115);
    bool in_short_rest = (minutes_in_clycle % 30 >= 25);
    return !in_long_rest && !in_short_rest;
}
\end{lstlisting}

\end{document}
