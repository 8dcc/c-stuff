\documentclass{amsart}

\title{Calculating pomodoro intervals}
\author{8dcc}

% Various math utilities, like align*
\usepackage{amsmath}

% Graphs
\usepackage{tikz}

% Link sections and subsections
\usepackage[hidelinks]{hyperref}
\hypersetup{linktoc=all}

% Code highlighting.
\usepackage{listings}
\lstset{
  % Showing spaces
  showspaces=false,
  showstringspaces=false,
  showtabs=false,
  % Indentation and breaks
  tabsize=4,
  breaklines=true,
  breakatwhitespace=true,
  columns=flexible,
  % Show left, right, top and bottom borders
  frame=tblr,
  % Misc
  aboveskip=3mm,
  belowskip=3mm,
  basicstyle={\small\ttfamily},
}

% Different monospace font for code blocks (listings)
\usepackage{inconsolata}

% Remove author and extra info from the headers
\pagestyle{plain}

% New environment adding spacing for tikz pictures
\newenvironment{tikzpicturecenter}
{\begin{center}\begin{tikzpicture}}
    {\end{tikzpicture}\end{center}}

\begin{document}
\maketitle

\section{The Pomodoro Technique}

The Pomodoro Technique is a time management method developed by Francesco
Cirillo in the late 1980s. It uses a kitchen timer to break work into intervals
(\textit{pomodoros}), typically 25 minutes in length, separated by short
breaks. After 4 pomodoros are done, a longer break is taken.

For this example, these will be the specifics timings:

\begin{enumerate}
\item Work for 25 minutes.
\item Rest for 5 minutes.
\item Go to step one, until 4 pomodoros have passed.
\item Once the 4 pomodoros are done, rest for 30 minutes.
\end{enumerate}

With this, we know that the first short break will start at minute 25 and end at
minute 30, and the first long break will start at minute 115 ($30 \cdot 4 - 5$)
and end at minute 145.

\section{The problem}

Given a number of minutes, how do we know if we should be resting or working
when that time has passed?

You can check if a minute is in a short or long break by calculating the modulus
of that minute with the end of the first break in the cycle, and then checking
if that value is greater or equal than the minute when that break started.

\begin{align*}
  \text{inShortBreak}(m) &= m \bmod 30 \geq 25 \\
  \text{inLongBreak}(m)  &= m \bmod 145 \geq 115
\end{align*}

However, the fifth pomodoro, the one after the first long break doesn't start
aligned to the 30 minute boundary ($145 \bmod 30 \neq 0$).

At first I thought about calculating the Least Common Multiple (LCM) of 145 and
30 (which is 870), since that calculates the end of a pomodoro that is after a
long break, while also being aligned with the end of a (supposed) short
break. However, that doesn't really simplify our problem.

\section{The solution}

The solution was actually quite simple. You can think of the break cycle as a
big 145-minute cycle, and calculate the minutes relative to that with a modulus
operation, just like we did before:

\begin{align*}
  \text{inShortBreak}(m) &= \left( m \bmod 145 \right) \bmod 30 \geq 25 \\
  \text{inLongBreak}(m)  &= m \bmod 145 \geq 115
\end{align*}

Since the end of long breaks is already aligned with the end of the cycles, the
\texttt{inLongBreak} function remains unchanged. The only change is calculating
the modulus one extra time in the \texttt{inShortBreak} function.

This is the C code for the previous functions:

\begin{lstlisting}[language=C]
bool inPomodoro(int minutes) {
    int minutes_in_clycle = minutes % ((25 + 5) * 3 + 25 + 30);
    bool in_long_break = (minutes_in_clycle >= 115);
    bool in_short_break = (minutes_in_clycle % 30 >= 25);
    return !in_long_break && !in_short_break;
}
\end{lstlisting}

\end{document}
