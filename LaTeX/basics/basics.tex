\documentclass{article}

\title{Scratch - Basics of \LaTeX}
\author{8dcc}
\date{2024}

% ------------------------------------------------------------------------------
% Packages
% ------------------------------------------------------------------------------

% Clickable table of contents
\usepackage[hidelinks]{hyperref}
\hypersetup{
  linktoc=all,     % Link sections and subsections
}

% Prevent indentation after lists, tables, etc.
\usepackage{noindentafter}
\NoIndentAfterEnv{itemize}
\NoIndentAfterEnv{enumerate}
\NoIndentAfterEnv{tabular}

% Code highlighting.
\usepackage{listings}
\lstset{
  % Showing spaces
  showspaces=false,
  showstringspaces=false,
  showtabs=false,
  % Indentation and breaks
  tabsize=4,
  breaklines=true,
  breakatwhitespace=true,
  columns=flexible,
  % Show left, right, top and bottom borders
  frame=tblr,
  % Misc
  aboveskip=3mm,
  belowskip=3mm,
  basicstyle={\small\ttfamily},
}

% Different monospace font for code blocks (listings)
\usepackage{inconsolata}

% ------------------------------------------------------------------------------
% Document start
% ------------------------------------------------------------------------------

\begin{document}

\maketitle
\newpage

\tableofcontents
\newpage

% ------------------------------------------------------------------------------
\section{Introduction}
% ------------------------------------------------------------------------------

Simple scratch file for testing the basics of \LaTeX.

% ------------------------------------------------------------------------------
\section{Text formats}
% ------------------------------------------------------------------------------

Here is a list of text formats:

\begin{itemize}
  \item \textbf{Bold}
  \item \textit{Italic}
  \item \underline{Underlined}
  \item \emph{Emphasized}
  \item \textsc{Small caps}
  \item \textrm{Roman}
  \item \textsl{Slanted}
  \item \texttt{Typewriter}
\end{itemize}

You can \underline{\textbf{\textit{combine}}} these effects too.

The underline distance changes depending on \underline{descenders} like the
\underline{letter p}. To avoid this, you can use the \verb|\smash| command. For
example: \underline{\smash{letter p}}.

To quote a \LaTeX command, you can use the \verb|verb| command. You can also use
\verb|\begin{verbatim}| and \verb|\end{verbatim}|.

This is a \href{http://github.com/8dcc}{\textbf{Link}}, thanks to the
``hyperref'' package.

% ------------------------------------------------------------------------------
\section{Lists}
% ------------------------------------------------------------------------------

This is an unordered list:

\begin{itemize}
  \item I am the first item.
  \item I am the second item.
\end{itemize}

This is an ordered list:

\begin{enumerate}
  \item I am the first item.
  \item I am the second item.
\end{enumerate}

This is some C code:

\begin{lstlisting}[language=C]
  while (a != b) {
    // ...
  }
\end{lstlisting}

% ------------------------------------------------------------------------------
\section{Lengths and spacing}
% ------------------------------------------------------------------------------

These are the different units that can be used in \LaTeX.

\vspace{1em}
\begin{tabular}{| p{0.1\textwidth} | p{0.75\textwidth} |}
  \hline
  Unit & Description \\
  \hline
  pt & a point is approximately 1/72.27 inch, that means about 0.0138 inch or
       0.3515 mm (exactly point is defined as 1/864 of American printer's foot
       that is 249/250 of English foot) \\
  mm & a millimeter \\
  cm & a centimeter \\
  in & inch \\
  ex & roughly the height of an `x' (lowercase) in the current font (it depends
       on the font used) \\
  em & roughly the width of an `M' (uppercase) in the current font (it depends
       on the font used) \\
  mu & math unit equal to 1/18 em, where em is taken from the math symbols
       family \\
  sp & so-called ``special points", a low-level unit of measure where
       65536sp=1pt \\
  \hline
\end{tabular}
\vspace{1em}

\noindent{}
These are the most common lengths that can be used in \LaTeX.

\vspace{1em}
\begin{tabular}{| p{0.25\textwidth} | p{0.6\textwidth} |}
  \hline
  Length & Description \\
  \hline
  \verb|\baselineskip|   & Vertical distance between lines in a paragraph \\
  \verb|\columnsep|      & Distance between columns \\
  \verb|\columnwidth|    & The width of a column \\
  \verb|\evensidemargin| & Margin of even pages, commonly used in two-sided
                           documents such as books \\
  \verb|\linewidth|      & Width of the line in the current environment. \\
  \verb|\oddsidemargin|  & Margin of odd pages, commonly used in two-sided
                           documents such as books \\
  \verb|\paperwidth|     & Width of the page \\
  \verb|\paperheight|    & Height of the page \\
  \verb|\parindent|      & Paragraph indentation \\
  \verb|\parskip|        & Vertical space between paragraphs \\
  \verb|\tabcolsep|      & Separation between columns in a table (tabular
                           environment) \\
  \verb|\textheight|     & Height of the text area in the page \\
  \verb|\textwidth|      & Width of the text area in the page \\
  \verb|\topmargin|      & Length of the top margin \\
  \hline
\end{tabular}

% ------------------------------------------------------------------------------
% Bibliography example

\newpage
\begin{thebibliography}{9}
\bibitem{wikipedia_example}
  Wikipedia. \emph{Example Reference}. Retrieved 1 Jan 2024, from
  \url{https://en.wikipedia.org/wiki/Example}
\bibitem{mathinsight_dotproduct}
  Nykamp DQ. \emph{The dot product}. From Math Insight. Retrieved 23 May 2024,
  from \url{https://mathinsight.org/dot_product}
\bibitem{reversing}
  Eldad Eilam. \emph{Reversing: Secrets of Reverse Engineering}.
  Wiley. 2005. pp. 56-57.
\end{thebibliography}

\end{document}
