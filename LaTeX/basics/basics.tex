\documentclass{article}

\title{Scratch - Basics of \LaTeX}
\author{8dcc}
\date{}

% Various math utilities, like \begin{align*}
\usepackage{amsmath}

\begin{document}
\maketitle

Simple scratch file for testing the basics of \LaTeX.

\section{Text types and lists}

You can make text \textbf{bold}, \textit{italic} and \underline{underlined}.
You can \underline{\textbf{\textit{combine}}} these effects too.

This is an unordered list.

\begin{itemize}
  \item I am the first item.
  \item I am the second item.
\end{itemize}

This is an ordered list.

\begin{enumerate}
  \item I am the first item.
  \item I am the second item.
\end{enumerate}

\section{Math}

The gravitational force of each body is calculated with the following formula.

\begin{displaymath}
    F = G \frac{m_1m_2}{r^2}
\end{displaymath}

Where \(G\) is the gravitational constant, \(m_1\) and \(m_2\) are the mass of
each body, and \(r\) is the distance between the objects.

The effect of a force is to accelerate the body. The relationship is the
following.

\begin{displaymath}
   F = m a
\end{displaymath}

Where \(F\) is the force, \(m\) is the mass and \(a\) is the acceleration of
the body. Therefore, to get the acceleration from the force, we can do the
following.

\begin{displaymath}
    a = \frac{F}{m}
\end{displaymath}

The force has a direction. It acts towards the direction of the line joining
the centres of the two bodies. We can get the X and Y coordinates of the
acceleration with some trigonometry.

\begin{align*}
    a_x &= a \cos(\theta) \\
    a_y &= a \sin(\theta) \\
\end{align*}

Where \(a_x\) and \(a_y\) are the X and Y accelerations, \(a\) is the
acceleration, and \(\theta\) is the angle that the line joining the centers make
with the horizontal.

\end{document}
